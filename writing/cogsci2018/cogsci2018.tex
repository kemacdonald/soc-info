% Template for Cogsci submission with R Markdown

% Stuff changed from original Markdown PLOS Template
\documentclass[10pt, letterpaper]{article}

\usepackage{cogsci}
\usepackage{pslatex}
\usepackage{float}
\usepackage{caption}

% amsmath package, useful for mathematical formulas
\usepackage{amsmath}

% amssymb package, useful for mathematical symbols
\usepackage{amssymb}

% hyperref package, useful for hyperlinks
\usepackage{hyperref}

% graphicx package, useful for including eps and pdf graphics
% include graphics with the command \includegraphics
\usepackage{graphicx}

% Sweave(-like)
\usepackage{fancyvrb}
\DefineVerbatimEnvironment{Sinput}{Verbatim}{fontshape=sl}
\DefineVerbatimEnvironment{Soutput}{Verbatim}{}
\DefineVerbatimEnvironment{Scode}{Verbatim}{fontshape=sl}
\newenvironment{Schunk}{}{}
\DefineVerbatimEnvironment{Code}{Verbatim}{}
\DefineVerbatimEnvironment{CodeInput}{Verbatim}{fontshape=sl}
\DefineVerbatimEnvironment{CodeOutput}{Verbatim}{}
\newenvironment{CodeChunk}{}{}

% cite package, to clean up citations in the main text. Do not remove.
\usepackage{cite}

\usepackage{color}

% Use doublespacing - comment out for single spacing
%\usepackage{setspace}
%\doublespacing


% % Text layout
% \topmargin 0.0cm
% \oddsidemargin 0.5cm
% \evensidemargin 0.5cm
% \textwidth 16cm
% \textheight 21cm

\title{Learning in social context}


\author{{\large \bf Erica J. Yoon*}, {\large \bf Kyle MacDonald*}, {\large \bf Mika Asaba}, {\large \bf Hyowon Gweon}, \and {\large \bf Michael C. Frank} \\ \{ejyoon, kylem4, masaba, hyo, mcfrank\} @stanford.edu \\ Department of Psychology, Stanford University \\ *These authors contributed equally to this work.}

\begin{document}

\maketitle

\begin{abstract}
\ldots{}

\textbf{Keywords:}
Learning; social context; information gain; OED; self-presentation; goal
tradeoff
\end{abstract}

\section{Introduction}\label{introduction}

Learning takes place in social contexts.

A simple case study of situations where people have to decide between
information gain and self-presentation,

\section{Computational model}\label{computational-model}

A key assumption underlying inferences in recent formal models of social
cognition is that people act approximately optimally given a utility
function (e.g. Goodman \& Frank, 2016; Jara-Ettinger, Gweon, Schulz, \&
Tenenbaum, 2016). Our model operates under the same principle of an
approximately optimal agent, who reasons about the utility function that
represents a combination of multiple goals. In Yoon, Tessler, Goodman,
\& Frank (2017), the utility function comprised a weighted combination
of multiple utilities (goals) considered by the speaker, reflecting a
principled tradeoff between different communicative goals (e.g.~to be
informative, to make the listener feel good, and to appear to be a
helpful speaker). We use a similarly structured utility function that
reflects different goals that a learner has in a social learning
context. Specifically, we model how a person may make a decision to act
based on his desire to learn how a toy works (\emph{learning utility}),
to make the toy operate and perform a given function (\emph{performance
utility}), or to present himself as a competent individual who knows how
to make the toy work (\emph{presentational utility}; see the model
diagram in Figure 1).

\begin{CodeChunk}
\captionsetup{width=0.8\textwidth}\begin{figure*}[h]

{\centering \includegraphics{figs/model_diagram-1} 

}

\caption[Diagram of the computational model]{Diagram of the computational model: The learner considers possible hypotheses and his contextual goals. When an observer is absent, he considers his learning goal (to maximize information gain) and performance goal (to play music) and decides on an action. When an observer is present, his decision for an action is based on his learning goal vs. presentational goal (to have the observer infer his competence).}\label{fig:model_diagram}
\end{figure*}
\end{CodeChunk}

First, the \emph{learning utility} symbolizes the goal to learn new
information, which in our paradigm specifically is associated with
figuring out how a given toy works. The learning utility is formally
represented by an OED model (Lindley, 1956; ``Optimal Experiment
Design''; Nelson, 2005), which quantifies the \emph{expected utility} of
different information seeking actions. Here we follow the mathematical
details of the OED approach as outlined in Coenen, Nelson, \& Gureckis
(2017) that was implemented in our model. The set of queries, each
realized through taking an action, is defined as
\(Q_1, Q_2, ..., Q_n = {Q}\). The expected utility of each query
(\(EU(Q)\)) is a function of two factors: (1) the probability of
obtaining a specific answer \(P(a)\) weighted by (2) the usefulness of
that answer for achieving the learning goal \(U(a)\).

\[EU(Q) = \sum_{a\in q}{P(a)U(a)}\]

There are a variety of ways to define the usefulness function to score
each answer (for a detailed analysis of different approaches, see Nelson
(2005)). One standard method is to use \emph{information gain}, which is
defined as the change in the learner's overall uncertainty (difference
in entropy) before and after receiving an answer. This information gain
is then the usefulness of the answer to the query, and thus is equal to
the learning utility:

\[ U_{learning} = U(a) = ent(H) - ent(H|a)\] \noindent
where \(ent(H)\) is defined using Shannon
entropy\footnote{Shannon entropy is a measure of unpredictability or amount of uncertainty in the learner's probability distribution over hypotheses. Intuitively, higher entropy distributions are more uncertain and harder to predict. For example, if the learner believes that all hypotheses are equally likely, then they are in a state of high uncertainty/entropy. In contrast, if the learner firmly believes in one hypothesis, then uncertainty/entropy is low.}.
MacKay (2003), which provides a measure of the overall amount of
uncertainty in the learner's beliefs about the candidate hypotheses.

\[ent(H) = -\sum_{a\in A}{P(h)log_2P(h)}\] \noindent
The conditional entropy computation is the same, but takes into account
the change in the learner's beliefs after seeing an answer.

\[ ent(H|a) = -\sum_{h\in H}{P(h|a)logP(h|a)} \] \noindent
To calculate the change in the learner's belief in a hypothesis
\(P(h|a)\), we use Bayes rule.

\[ P(h|a) = \frac{P(h)P(a|h)}{P(a)} \]

\noindent
The learner performs the expected utility computation for each query in
the set of possible queries and picks the one that maximizes utility. In
practice, the learner considers each possible answer, scores the answer
with the usefulness function, and weights the score using the
probability of getting that answer. In our paradigm, a learner thinking
about the learning utility considers acting on the toy one way over
another, and computes how informative a given answer should be in
reducing uncertainty about how the toy works.

Second, the \emph{performance utility} is the utility of successfully
making the toy operate. Specifically within our current paradigm, the
performance utility is the expected utility of music playing (\(m\))
given the learner's action \(a\).

\[ U_{performance} = \ln(P_L(m | a)) \] \noindent
Thus, performance utility is maximized by taking an action that is most
likely to make the toy ``go'' and play music, which is the operation of
interest.

When there is no observer present, the learner considers the tradeoff
between the learning utility and performance utility, and he determines
his action based on a weighted combination of the two utilities:

\[ U(a;\phi; obs = no) = \phi \cdot U_{learning} + (1-\phi) \cdot U_{performance} ,\]
\noindent
where \(\phi\) is a mixture parameter governing the extent to which the
learner prioritizes information gain over making the toy play music.

When there is another person present to observe the learner's action,
this observer \(O\) reasons about the competence \(c\) of the learner
\(L\) which is equal to whether the learner was able to make the toy
work.

\[ P_O(c) \propto P_L(m | a)\]

The learner thinks about how the observer infers the learner's
competence, and his \emph{presentational} utility is based on maximizing
the apparent competence inferred by the observer.

\[ U_{presentation} = \ln(P_O(c)) \] Thus, when there is an observer
present, the learner considers the tradeoff between the learning utility
and presentational utility:

\[ U(m;a;\phi; obs = yes) = \phi \cdot U_{learning} + (1-\phi) \cdot U_{presentational}\]
Based on the utility functions above, the learner (\(L\)) chooses his
action \(a\) approximately optimally (as per optimality parameter
\(\lambda\)) given his goal weight and observer presence.

\[ P_L(a | \phi, obs) \propto \exp(\lambda \cdot \mathbb{E}[U(a;\phi; obs)])\]

\section{Experiment}\label{experiment}

\subsection{Method}\label{method}

\subsubsection{Participants}\label{participants}

FIXME participants with IP addresses in the United States were recruited
on Amazon's Mechanical Turk. We excluded FIXME partcipants who failed to
answer manipulation check questions correctly (See Procedure section for
details on the manipulation check questions), and thus the remaining
FIXME participants were included in our final analysis.

\subsubsection{Stimuli and Design}\label{stimuli-and-design}

We asked participants to imagine they were children's toy developers. We
presented three different toys that look very similar but each work in
different ways, and provided instructions for them. \emph{The
ButtonMusic toy} instructions were: \emph{``Press the button on the
right to play music. Pull the handle on the left to turn on the light.
Doing both produces both effects.''} ``\emph{The HandleMusic toy}
instuctions were: \emph{``Pull the handle on the left to play music.
Press the button on the right to turn on the light. Doing both produces
both effects.''} and''\emph{the BothMusicLight toy} instuctions were:
\emph{``Pull the handle on the left AND press the button on the right to
turn on the light and play music at the same time. The button press or
handle pull on its own doesn't produce any effect.''} Each toy had a
label showing its name.

We presented a story to the participants that their boss motivated a
goal the participants must achieve by acting on the toy. Importantly,
the toy was missing its label, such that partcipants could not know
whether the toy was a ButtonMusic, HandleMusic, or BothMusicLight toy.
In the \emph{learning} condition, the boss said ``That must be one of
the new toys that you've been working on. But it looks like you forgot
to put on the label! Can you figure out whether this toy is a
ButtonMusic toy, HandleMusic toy, or BothMusicLight toy?''; in the
\emph{performance} condition, the boss said ``That must be one of the
new toys that you've been working on. I want to hear the music it
plays.''; and in the \emph{presentation} condition, the boss said ``That
must be one of the new toys that you've been working on. How does it
work?'' followed by the prompt ``\ldots{} you only had one chance to
impress your boss and show that you're competent \ldots{}'' Then we
asked participants to select one action to try out on the toy: to
``press the button'', ``pull the handle'', or ``press the button and
pull the handle.'' Each participant was randomly assigned to one of
three goal conditions, and shown a randomized order of actions to choose
from.

\subsubsection{Procedure}\label{procedure}

Participants were first introduced to the task and shown a picture of a
toy with labels on its parts. Then they read instructions for each of
the three toys, after which they were asked what they would do to make
the toy play music and to make it turn on the light, to make sure they
understood the instructions. Participants who answered these
manipulation check questions incorrectly were excluded from our final
analysis. We then asked participants to rate prior likelihood that an
unknown toy is a ButtonMusic toy, HandleMusic toy, or BothMusicLight
toy. Participants read a scenario for one of the three goal conditions,
and the following instruction: ``If you only had one chance to try a
SINGLE action to {[}goal{]}, which action would you want to take? You
will get a 10 cent bonus after submitting the HIT if you {[}goal{]}.''
After selecting a response out of three possible actions, the
participants were asked again to rate the likelihood for which toy the
unlabeled toy was. The experiment can be viewed at
\url{https://langcog.stanford.edu/expts/EJY/soc-info/goal_actions_ver2/soc_info_goals.html}.

\section{Results}\label{results}

\section{Discussion}\label{discussion}

\section{Acknowledgements}\label{acknowledgements}

This work was supported by NSERC postgraduate doctoral scholarship
PGSD3-454094-2014 to EJY \ldots{} FIXME.

\section{References}\label{references}

\setlength{\parindent}{-0.1in} \setlength{\leftskip}{0.125in} \noindent

\hypertarget{refs}{}
\hypertarget{ref-coenen2017}{}
Coenen, A., Nelson, J. D., \& Gureckis, T. (2017). Asking the right
questions about human inquiry.

\hypertarget{ref-goodman2016}{}
Goodman, N. D., \& Frank, M. C. (2016). Pragmatic language
interpretation as probabilistic inference. \emph{Trends in Cognitive
Sciences}, \emph{20}(11), 818--829.

\hypertarget{ref-jara2016}{}
Jara-Ettinger, J., Gweon, H., Schulz, L. E., \& Tenenbaum, J. B. (2016).
The naïve utility calculus: Computational principles underlying
commonsense psychology. \emph{Trends in Cognitive Sciences},
\emph{20}(8), 589--604.

\hypertarget{ref-lindley1956}{}
Lindley, D. V. (1956). On a measure of the information provided by an
experiment. \emph{The Annals of Mathematical Statistics}, 986--1005.

\hypertarget{ref-mackay2003}{}
MacKay, D. J. (2003). \emph{Information theory, inference and learning
algorithms}. Cambridge university press.

\hypertarget{ref-nelson2005}{}
Nelson, J. D. (2005). Finding useful questions: On bayesian
diagnosticity, probability, impact, and information gain.
\emph{Psychological Review}, \emph{112}(4).

\hypertarget{ref-yoon2017}{}
Yoon, E. J., Tessler, M. H., Goodman, N. D., \& Frank, M. C. (2017). ``I
won't lie, it wasn't amazing'': Modeling polite indirect speech. In
\emph{Proceedings of the thirty-ninth annual conference of the Cognitive
Science Society}.

\end{document}
