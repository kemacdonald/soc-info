% Template for Cogsci submission with R Markdown

% Stuff changed from original Markdown PLOS Template
\documentclass[10pt, letterpaper]{article}

\usepackage{cogsci}
\usepackage{pslatex}
\usepackage{float}
\usepackage{caption}

% amsmath package, useful for mathematical formulas
\usepackage{amsmath}

% amssymb package, useful for mathematical symbols
\usepackage{amssymb}

% hyperref package, useful for hyperlinks
\usepackage{hyperref}

% graphicx package, useful for including eps and pdf graphics
% include graphics with the command \includegraphics
\usepackage{graphicx}

% Sweave(-like)
\usepackage{fancyvrb}
\DefineVerbatimEnvironment{Sinput}{Verbatim}{fontshape=sl}
\DefineVerbatimEnvironment{Soutput}{Verbatim}{}
\DefineVerbatimEnvironment{Scode}{Verbatim}{fontshape=sl}
\newenvironment{Schunk}{}{}
\DefineVerbatimEnvironment{Code}{Verbatim}{}
\DefineVerbatimEnvironment{CodeInput}{Verbatim}{fontshape=sl}
\DefineVerbatimEnvironment{CodeOutput}{Verbatim}{}
\newenvironment{CodeChunk}{}{}

% cite package, to clean up citations in the main text. Do not remove.
\usepackage{cite}

\usepackage{color}

% Use doublespacing - comment out for single spacing
%\usepackage{setspace}
%\doublespacing


% % Text layout
% \topmargin 0.0cm
% \oddsidemargin 0.5cm
% \evensidemargin 0.5cm
% \textwidth 16cm
% \textheight 21cm

\title{Balancing informational and social goals in active learning}


\author{{\large \bf Erica J. Yoon*}, {\large \bf Kyle MacDonald*}, {\large \bf Mika Asaba}, {\large \bf Hyowon Gweon}, \and {\large \bf Michael C. Frank} \\ \{ejyoon, kylem4, masaba, hyo, mcfrank\} @stanford.edu \\ Department of Psychology, Stanford University \\ *These authors contributed equally to this work.}

\begin{document}

\maketitle

\begin{abstract}
Our actions shape what we learn. Recent work suggests that people engage
in efficient self-directed learning to maximize information gain.
However, human learning often unfolds in social contexts where learners
not only face informational goals (e.g.~learn how something works) but
also social goals (e.g.~appear competent and impress others). How do
these factors shape learners' decisions? Here, we present a
computational model that integrates the value of social and information
goals to predict the decisions that people will make in a simple active
causal learning task. We show that an emphasis on performance or
self-presentation goals leads to reduced chances of learning (E1). Next,
we show that social context can push learners to pursue
performance-oriented actions even when the learning goal is highlighted
(E2). Our formal model of social-active learning successfully captures
the empirical results. These findings are the first steps towards
understanding the role of social reasoning in active learning contexts.

\textbf{Keywords:}
active learning; social reasoning; information gain; OED;
self-presentation; goal tradeoffs
\end{abstract}

\section{Introduction}\label{introduction}

Imagine you are a novice cook and you have to decide what meal to
prepare for a first date. Should you choose an easy favorite or should
you attempt to make something new? While the familiar recipe can ensure
a good meal, you may miss out on a new, delicious dish. The new recipe
might taste even better, but it has a higher chance of failure.

We often have to choose between \emph{exploration} and
\emph{exploitation}: that is, actions that could (a) lead to an overt,
readily accessible reward based on what we already know
(\emph{exploitation}) or (b) result in the discovery of new information
(\emph{exploration}; Sutton \& Barto, 1998). This decision of whether to
explore or exploit is directly related to the relative strength of the
goals within a particular context. In the cooking example, I can
prioritize the goal of learning by cooking the new recipe, or I can
instead focus on the performance goal by preparing the tried and true
meal. Here, we explore the idea of this learning-performance goal
tradeoff in a simple active learning context, where social factors may
shape the goals we consider.

\emph{Active learning} refers to situations where people are given
control over the sequence of information in a learning context (e.g.~try
pressing different buttons on a toy, one by one, to see if it produces
an effect). The key assumption is that learners will maximize the
usefulness of their actions by gathering information that is especially
helpful for their own learning. The effects of active learning have been
the focus of much empirical work in education (Grabinger \& Dunlap,
1995), machine learning (Settles, 2012), and cognitive psychology
(Castro et al., 2009), with the common finding that active contexts lead
to faster learning than passive contexts where people don't have control
over the information flow.

But real-world learning usually takes place in rich social contexts with
teachers, peer learners, or other people who can directly influence our
learning. Indeed, it has been suggested that children and adults
modulate their inferences depending on whether they generate their own
evidence, or learn from evidence generated by others (e.g. Xu \&
Tenenbaum, 2007). Other work suggests that children learn faster when
observing intentional (more informative) actions compared to accidental
(less informative) actions (Carpenter, Akhtar, \& Tomasello, 1998).
Moreover, adults and children will make even stronger inferences if they
believe that another person selected their actions with the goal of
helping them learn (i.e.~teaching; Shafto, Goodman, \& Frank, 2012).

However, social influences are not only present when we learn from
others. Even when we learn from our own actions, our social environment
may affect our self-directed learning process. While previous models
have captured how we optimize learning, either from our own actions or
from others, they have been agnostic to other social factors that are
ubiquitous in a learner's environment. People must integrate the value
of social goals (e.g.~looking competent or knowledgeable) and
information goals when deciding what to do next. Moreover, actions that
maximize learning are inherently risky in that you can potentially fail
to produce an immediate outcome, and thus may be more difficult to
undertake with someone else present who might judge you as incompetent.

How can active learning models accommodate this richer set of utilities?
As a step towards answering this question, we model a learner who
considers a mixture of learning and performance goals. A key assumption
underlying inferences in recent Bayesian models of human social
cognition is that people expect others to act approximately optimally
given a utility function (e.g. Goodman \& Frank, 2016; Jara-Ettinger,
Gweon, Schulz, \& Tenenbaum, 2016). Our model adopts the same
utility-theoretic approach, and assumes an approximately optimal agent
who reasons about a utility function that represents a weighted
combination of multiple goals (Yoon, Tessler, Goodman, \& Frank, 2017).
Our model thus reflects a tradeoff between different goals that a
learner has in a social learning context.

We instantiate our model in a simple causal learning task and examine
whether people choose actions that support learning vs.~social goals. We
present a toy with an uncertain causal mechanism (Figure \ref{fig:toy}),
for which doing only one of the two possible actions (handle pull or
button press) is disambiguating but potentially risks no immediate
effect (i.e.~neither sound nor light turning on), while doing both
actions simultaneously is immediately rewarding but is not informative
for learning the toy's causal mechanism. Thus, the learner can choose
between the two actions that will each lead to one outcome (new
discovery) or the other (immediate reward). The learner's action rests
on relative utilities he assigns to exploration versus exploitation,
which in turn are determined in part by the presence or absence of
another person he cares about (i.e.~his
boss).\footnote{From here on, we use a male pronoun for Bob, the learner, and female pronoun for Ann, the boss and observer.}

In two experiments, we show that emphasizing performance or
self-presentation goals leads to actions that are not informative and
thus reduce the chances of learning (E1). Next, we show that the
presence of an observer (i.e., a boss) pushes learners to pursue
performance/presentation actions even when the learning goal is
highlighted (E2). Finally, we present a Bayesian Data Analysis showing
that the empirical results are consistent with predictions of our
cognitive model of social-active learning.

\section{Computational model}\label{computational-model}

\begin{CodeChunk}
\begin{figure}[t]

{\centering \includegraphics[width=0.65\linewidth]{figs/toy-1} 

}

\caption[An example of the toy used in our paradigm]{An example of the toy used in our paradigm.}\label{fig:toy}
\end{figure}
\end{CodeChunk}

\begin{CodeChunk}
\begin{figure*}[tb]

{\centering \includegraphics[width=0.95\linewidth]{figs/model_diagram-1} 

}

\caption[Diagram of the computational model]{Diagram of the computational model. The learner considers possible hypotheses: Toy 1 (handle pull turns on the light, button press turns on music, both actions cause both effects); Toy 2 (handle pull turns on music, button press turns on the light, both actions cause both effects); and Toy 3 (both actions cause both effects, but each action on its own does not produce any effect). The learner also considers his contextual goals. When an observer is absent, he considers his learning goal (to maximize information gain) and performance goal (e.g. to play music) and decides on an action. The learning goal favors a single action (e.g. pull the handle only) that can fully disambiguate, whereas the performance goal favors the both action (pull the handle AND push the button) that guarantees the most salient reward. When an observer is present, his decision for an action is based on his learning goal vs. presentational goal (to have the observer infer his competence or knowledge of how the toy works).}\label{fig:model_diagram}
\end{figure*}
\end{CodeChunk}

We model a learner's action choice based on his desire to learn how a
toy works (\emph{learning utility}), to make the toy operate and perform
a given function (\emph{performance utility}), or to present himself as
a competent individual who knows how to make the toy work
(\emph{presentational utility}; see the model diagram in Figure
\ref{fig:model_diagram}).

\subsubsection{Learning utility}\label{learning-utility}

The \emph{learning utility} symbolizes the goal to learn new
information, which in our paradigm is associated with figuring out how a
given toy works. The learning utility is formally represented by an OED
model (Lindley, 1956; ``Optimal Experiment Design''; Nelson, 2005),
which quantifies the \emph{expected utility} of different information
seeking actions. Here we follow the mathematical details of the OED
approach as outlined in Coenen, Nelson, \& Gureckis (2017). The learner
considers the hypothesis space \(H\), and wants to determine the correct
hypothesis. Based on a set of queries, each realized through taking an
action, the learner thinks about the utility of the answer to each
query. The utility of answer is equal to the \emph{information gain},
which is the change in the learner's overall uncertainty (difference in
entropy) before and after receiving an answer. This information gain is
then the usefulness of the answer to the query, and thus is equal to the
learning utility (\(U_{learn}\)):
\[ U_{learn} = U(a) = \frac{ent(H) - ent(H|a)}{log_2n}\] \noindent
where \(ent(H)\) is the Shannon entropy of \(H\), which provides a
measure of the overall amount of uncertainty in the learner's beliefs
about the candidate hypothesis(MacKay, 2003). Once the learner chooses a
query \(Q\), which yields an answer \(a\), then he updates his beliefs
about each hypothesis via standard Bayesian updating. Finally, the
difference in entropy is normalized by \(log_2 n\), where \(n\) is the
number of possible actions, to generate a value between 0 and 1.

\subsubsection{Performance utility}\label{performance-utility}

The \emph{performance utility} is the utility of successfully making the
toy operate and achieving an immediate rewarding outcome. Within our
paradigm, the learner gains utility by seeing an immediate effect of
music or light turning on. The expected performance utility
(\(U_{perf}\)) before the learner chooses an action is then the
likelihood of an effect \(m\) given the learner's action \(a\). Thus,
the performance utility is maximized when the toy is expected to ``go.''
\[ U_{perf} = P_L(m | a) \] \noindent

When there is no observer present (\(obs = no\)), the learner considers
the tradeoff between the learning utility and performance utility, and
he determines his action based on a weighted combination of the two
utilities:
\[ U(\phi; obs = no) = \phi_{learn} \cdot U_{learn} + \phi_{perf} \cdot U_{perf} ,\]
\noindent
where \(\phi\) is a mixture parameter governing the extent to which the
learner prioritizes learning over performance.

\subsubsection{Presentation utility}\label{presentation-utility}

When there is another person present to observe the learner's action,
this observer \(O\) is expected to reason about the competence \(c\) of
the learner \(L\) which is equal to whether the learner was able to make
the toy produce an effect. The learner thinks the observer's inferential
process, and the expected \emph{presentational} utility (\(U_{pres}\))
is based on maximizing the apparent competence inferred by the observer.
\[ P_O(c) \propto P_L(m | a)\] \[ U_{pres} = P_O(c)\] When there is an
observer present (\(obs = yes\)), the learner considers the tradeoff
between all three utilities: the learning utility, performance utility
and presentational utility:
\[ U(\phi; obs = yes) = \phi_{learn} \cdot U_{learn} + \\ \phi_{perf} \cdot U_{perf} + \phi_{pres} \cdot U_{pres}\]
The learner \(L\) chooses his action \(a\) approximately optimally (as
per optimality \(\lambda\)) based on the expected utility given his goal
weights and observer presence.
\[ P_L(a | \phi, obs) \propto \exp(\lambda \cdot \mathbb{E}[U(\phi; obs)])\]

\section{Experiment 1}\label{experiment-1}

In Experiment 1 (E1), we first wanted to confirm that participants would
choose different actions depending on what goal was highlighted. We were
also interested in how people would act when no explicit goal was
specified within the task. Participants were asked to imagine that they
needed to act on a toy with an uncertain causal mechanism, and were
assigned to different goal conditions: (1) learning (learn how the toy
works), (2) performance (make the toy play music), (3) presentation
(impress their boss), and (4) no goal specified. We hypothesized that
participants would choose an informative action more often in the
following order of goal conditions (decreasing): learning, no goal,
performance, and
presentation.\footnote{Our hypothesis, method, model and data analysis were pre-registered prior to data collection on the Open Science Framework (https://osf.io/kcjau). All experiments, data, model scripts, and analysis codes for the statistical models can be found in the online repository for this project: https://github.com/kemacdonald/soc-info.}

\begin{CodeChunk}
\begin{figure*}[tb]

{\centering \includegraphics[width=0.95\linewidth]{figs/e1_behav_results_plot-1} 

}

\caption[Behavioral results for E1]{Behavioral results for E1. Panel A shows the proportion of action decisions for each goal condition. Error bars represent 95\% binomial confidence intervals computed using a Bayesian beta-binomial model. Panel B shows violin plots of participants' response times on the action decisions. Each point represents a participant with the width of the violin representing the density of the data at that value. Panel C shows violin plots of participants' belief change (entropy; information gain in bits) as a function of condition. Lower values represent higher certainty after selecting an action. Color in panels A and C represent the type of action participants selected.}\label{fig:e1_behav_results_plot}
\end{figure*}
\end{CodeChunk}

\subsection{Method}\label{method}

\subsubsection{Participants}\label{participants}

We recruited 196 participants (45-51 per condition) on Amazon's
Mechanical Turk, with IP addresses in the United States and a task
approval rate above 85\%. We excluded 7 participants who failed to
answer at least two out of three manipulation check questions correctly
(see Procedure section for details on the manipulation check), and thus
the remaining 189 participants were included in our final analysis.

\subsubsection{Stimuli and Design}\label{stimuli-and-design}

We presented images and instructions for three different toys that
looked very similar but worked in different ways (see captions for
Figure 2 \ref{fig:model_diagram}). The instructions communicated that
pressing the button and pulling the handle was immediately rewarding but
uninformative (fails to disambiguate the causal mechanism). In contrast,
either of the single actions was completely disambiguating, but was
uncertain to produce an immediate outcome. Each toy had a label at the
front, indicating the correct action(s)--outcome link.

We asked participants to act on one of these toys; importantly, the
given toy was missing its label, leading to uncertainty about its causal
structure. We randomly assigned participants into four goal conditions.
In the \emph{No-Goal} condition participants selected actions without
any goal specified. In the \emph{Learning}, \emph{Performance}, and
\emph{Presentation} conditions, we asked participants to imagine that
they were children's toy developers and one day their boss approached
them. We then instructed participants to: figure out the correct label
for the toy (\emph{Learning}); make the toy play music (or turn the
light on; \emph{Performance}); or impress their boss and show that they
are competent (\emph{Presentation}). We asked participants to select an
action out of the following set: ``press the button'', ``pull the
handle'', or ``press the button and pull the handle.'' The order of
actions was randomized.

\subsubsection{Procedure}\label{procedure}

In the \emph{exposure phase}, we showed participants an example toy and
gave instructions for three toy types. We first presented the
instructions for the single action toys (Toy 1 and Toy 2) in a
randomized order, and then presented the instructions for the both
action toy (Toy 3). After instructions, participants indicated what
action would make each toy operate (e.g. ``How would you make {[}this{]}
toy play music?'') to show that they understood how the different toys
worked.

In the \emph{test phase}, participants read a scenario for one of the
three goal conditions, followed by the question: ``If you only had one
chance to try a SINGLE action to {[}pursue the specified goal{]}, which
action would you want to take? You will get a 10 cent bonus \ldots{} if
you {[}achieve the given goal{]}''.

Both before and after the critical action decision trial, we asked
participants to rate the likelihood that an unknown toy was Toy 1, 2, or
3 by adjusting a continuous slider bar which ranged on a scale from
``very unlikely'' to ``very likely.'' These measurements indexed
participants' prior beliefs over hypotheses about how the toys were
likely to function and their \emph{belief change} after selecting an
action and observing its effect.

\subsection{Results and discussion}\label{results-and-discussion}

\subsubsection{Analysis plan}\label{analysis-plan}

We present behavioral analyses of participants' (1) action decisions,
(2) action decision times, and (3) belief change (i.e., learning).
Decision times correspond to the latency to make an action selection as
measured from the start of the action decision trial (all RTs were
analyzed in log space). We quantified participants' beliefs about the
possible toy designs using entropy, and belief change was measured as
the difference in entropy before and after selecting an action.

We used the \texttt{rstanarm} (Gabry \& Goodrich, 2016) package to fit
Bayesian regression models estimating the differences across conditions.
We report the uncertainty in our point estimates using 95\% Highest
Density Intervals (HDI). The HDI provides a range of credible values
given the data and model.

\subsubsection{Action decisions:}\label{action-decisions}

We modeled action decisions using a logistic regression specified as
\texttt{$action \sim goal\_condition$} with the No-Goal condition as the
reference category. Participants' tendency to select a ``single'' action
varied across conditions in the predicted pattern (see Panel A of Figure
\ref{fig:e1_behav_results_plot}), with the highest proportion occuring
in the Learning context, followed by the No Goal context, then
Performance, and the fewest single actions in the Presentation
condition.

Compared to the No-Goal condition, participants selected the single
action at a greater rate in the Learning condition (\(\beta\) = 1.28,
{[}0.5, 2.17{]}) and at lower rate in the Presentation context
(\(\beta\) = -1.41, {[}-2.47, -0.4{]}), with the null value of zero
difference condition falling well outside the 95\% HDI, and at similar
rate in the Performance condition (\(\beta\) = -0.53, {[}-1.43, 0.35{]})
with the 95\% HDI including the null.

TODO\_KM: add pairwise comparisons of preformance/presentation and
performance/no-goal.

\subsubsection{Action decision times:}\label{action-decision-times}

We analyzed response times in log space using the same model
specification. Figure \ref{fig:e1_behav_results_plot}A shows the full RT
data distribution. Compared to the No-Goal condition (\(M\)= 31
seconds), participants took on on average 12.2 (4.2, 20) seconds longer
to generate a decision in the Learning condition. In contrast,
participants in the Performance and Presentation conditions produced
similar decision times.

\subsubsection{Belief change:}\label{belief-change}

We modeled change in entropy as a function of goal condition and
participants' action choices:
\texttt{$entropy\_change \sim goal\_condition + action\_response$} (see
Figure \ref{fig:e1_behav_results_plot}C). Across all conditions, people
who selected the single action showed a greater reduction in entropy
(\(\beta\) = -0.49, {[}-0.64, -0.33{]}, i.e., learned more from their
action. We did not see evidence of an interaction between goal condition
and action selection. However, recall that a larger proportion of
participants selected the single action in the Learning context, so the
probability of learning is higher in this scenario.

\begin{CodeChunk}
\begin{figure*}[tb]

{\centering \includegraphics[width=0.95\linewidth]{figs/e2_results-1} 

}

\caption[Behavioral and model fitting results for E2]{Behavioral and model fitting results for E2. Panel A shows actions decisions with color representing social context, from human data (top) and fitted model predictions (bottom). Panel B shows decision times. Panel C shows belief change. Panel D shows inferred phi values for each goal-context condition. All other plotting conventions are the same as Figure 3.}\label{fig:e2_results}
\end{figure*}
\end{CodeChunk}

\section{Experiment 2}\label{experiment-2}

In E1, we confirmed that participants selected different actions
depending on the type of goal emphasized by the social context. In E2,
our goals were three-fold: (1) to replicate the results from E1; (2) to
manipulate goals \emph{and} the presence/absence of another person
indepently, allowing us to measure the interaction between goals and
social context; and (3) to compare empirical data with predictions of
our computational model. Our key behavioral prediction was an
interaction: that participants would be less likely to select the single
(more informative) action in the Learning goal and No-Goal conditions
when their boss was present. We also predicted a null result: that the
presence of the boss should not affect action decisions in the
Performance condition.

\subsection{Method}\label{method-1}

\subsubsection{Participants}\label{participants-1}

Using the same recruitment and exclusion criteria as E1, we recruited
347 participants (42-51 per condition), and 325 participants were
included in our final analysis.

\subsubsection{Stimuli and Design}\label{stimuli-and-design-1}

The stimuli and design were identical to E1, except we had seven
different goal \(\times\) social conditions. Goals remained identical to
E1; social context varied depending on whether the boss was present in
the cover story (\emph{social}) or she was absent (\emph{no-social}).
Thus, the seven conditions were: \emph{social-learning},
\emph{social-performance}, \emph{social-presentation},
\emph{no-social-no-goal}, \emph{no-social-learning},
\emph{no-social-performance}, and \emph{social-no-goal}. Note that we
did not have \emph{no-social-presentation} condition, because the
presentation goal was defined by presenting oneself as competent to
another person.\\
\#\#\# Procedure

The procedure was identical to E1.

\subsection{Results and discussion}\label{results-and-discussion-1}

\subsubsection{Action decisions:}\label{action-decisions-1}

We modeled action decisions using a logistic regression specified as
\texttt{$action \sim goal\_condition * social\_context$} with the
no-goal-no-social condition as the reference category. We replicated the
key finding from E1: participants selected a ``single'' action more
often when they were in a context that emphasized a learning goal,
followed by the no-goal, performance, and presentation conditions (see
panel A Fig 4). There was a main effect of social context, with
participants being less likely to select the single action when their
boss was present (\(\beta =\) -0.521, {[}-1.005, -0.053{]}). Finally,
there was evidence for a reliable interaction between goal condition and
social context such that the effect of social context was present in the
Learning and No-Goal conditions, but not in the Performance condition
(\(\beta\) \(_{int}\) = 1.163, {[}0.01, 2.312{]}).

\subsubsection{Action decision times:}\label{action-decision-times-1}

We replicated the key decision time finding from E1, with participants
making slower decisions in the Learning context as compared to
Performance/Presentation. On average, participants took seconds to
generate a response in the No-goal condidition and seconds in the
Learning condition. In contrast, decisions were faster in the
Performance (\(\beta\) = -7.78 sec, {[}-14.01, -1.52{]}) and
Presentation (-10.77 seconds, {[}-18.67, -2.73{]}) conditions, which
were similar to one another (see Panel B of Fig \ref{fig:e2_results}).
There was no evidence of a main effect of social context or an
interaction between goal condition and social context. Note that here we
did not see a difference in decision times between the Learning and
No-Goal conditions, which is different from the pattern in E1.

\subsubsection{Belief change:}\label{belief-change-1}

Across all conditions, participants who selected the single action
showed a greater reduction in entropy (\(\beta\) = -0.35, {[}-0.45,
-0.24{]}. There was some (weaker) evidence of greater reduction in
entropy in the Learning goal condition (\(\beta\) = -0.12, {[}-0.25,
0.01). There was no evidence of a main effect of social context and no
two- or three-way interactions between social context, goal condition,
and type of action choice.

\subsubsection{BDA model-data fit:}\label{bda-model-data-fit}

In our paradigm, participants were instructed to choose an
action\footnote{For action priors, we used a separate prior elicitation task, in which people indicated the likelihood for selecting an action without any background information about possible hypotheses or goals. The results suggested that none of the action choice priors statistically differed from chance. We used mean likelihood for each action choice as baseline priors in our model.}
based on a certain goal. We assumed that the goal descriptions (e.g.
``figure out the correct label for the toy'') conveyed to the
participants a particular set of goal weights \{\(\phi_{learn}\),
\(\phi_{perf}\), \(\phi_{pres}\)\} that they used to generate their
action choices. We put uninformative priors on these weights
(\(\phi \sim Uniform(0,1)\)) and inferred their credible values
separately for each pair of different goal condition and social context,
using Bayesian data analytic techniques (Lee \& Wagenmakers, 2014).

The inferred goal weights were consistent with what we predicted (see
Figure \ref{fig:e2_results}, panel D). \(\phi_{learn}\) was at its
highest for no-social learning condition, in which the goal to learn was
highlighted, and there was minimum social pressure. On the other hand,
the \(\phi_{perf}\) and \(\phi_{pres}\) together make up the highest
portion in the presentation condition, with high social pressure to
present competence, compared to other conditions.

We also inferred another parameter of the cognitive model, the
optimality parameter \(\lambda\). We put uninformative prior on the
parameter (\(\lambda \sim Uniform(0,10)\) and inferred its posterior
credible value from the data. We ran 4 MCMC chains for 100,000
iterations, discarding the first 50,000 for burnin. The Maximum A-
Posteriori (MAP) estimate and 95\% Highest Probability Density Interval
(HDI) for \(\lambda\) was 4.79 {[}3.96, 6.2{]}.

The predictions of the action choices according to the fitted learner
model are shown in Figure \ref{fig:e2_results}, panel A (bottom). The
model's expected posteriors over action choices capture key differences
between conditions: the single action was more likely for no-social than
social conditions overall, but not when the performance goal was
highlighted. The model was able to predict the distribution of action
responses with high accuracy \(r^2(21)\) = 0.9.

\section{General Discussion}\label{general-discussion}

How do social contexts shape active learning? Here, we proposed that
people integrate learning-, performance-, and presentation-oriented
goals when deciding what to do next. In two experiments, people's we
showed that people chose more informative actions when learning goals
were highlighted and in the absence of a relevant social context (i.e.,
no boss present), while they chose more immediately rewarding actions
when performance or presentational goals were highlighted, especially
when a boss was present. When no explicit goal was specified, people
showed behavior that seemed to reflect a mixture of goals. Our model of
social-active learning successfully captured key patterns in the
behavioral data.

This work represents a way to bring active learning accounts into
contact with social learning theories. We used ideas from Optimal
Experiment Design, which models active learning as a process of rational
choice that maximizes utility withrespect to gaining information, and
Rational Speech Act theory, which formalizes a process of recursive
social reasoning in language use. This step allowed us to include social
information within a formal utility-theoretic framework, building a
richer utility function that represented a weighted combination of
multiple goals -- both informational and social.

There are important limitations to this work that present opportunities
for future work. First, we did not differentiate between performance and
presentation goals in the current model/pardigm. That is, the choice of
doing both actions satisfies both performance and presentational gaols.
Our future work is aimed at enriching the space of actions that people
could take, which should allow us to tease apart actions driven
self-presentation. Second, we used a very particular social context --
the presence of a boss -- to influence people's action choices. Thus, it
remains an open question as to how these results would generalize to
other kinds of observers that hold different goals. One particularly
compelling contrast would be to a teacher who wants the learner to
select actions that help her learn. Third, we limited people to a single
action choice. While this allowed us to get a clean measurement of our
goal and social context manipulations, real-world learning often
involves a process of sequential decision-making that could cause
learners to priortize different goals depending on their own prior
actions and/or the proability of interacting with an observer in the
future.

Another interesting open question is how our model/results could be used
to understand active learning over development. Our framework would in
principle allow us to measure changes in children's goal preferences as
they develop more sophisticated social reasoning and better
meta-cognitive abilities. One prediction is that young children focus on
learning goals earlier in development when they are surrounded by
familiar caregivers who scaffold learning-relevant actions. But as their
social reasoning abilities mature and their social environments become
more complex, children may start to emphasize performance or
presentation goals.

Overall, this work represents a first step to answering these rich
questions that ultimately seek to unify theories on active learning and
social reasoning.

\section{Acknowledgements}\label{acknowledgements}

This work was supported by NSERC postgraduate doctoral scholarship
PGSD3-454094-2014 to EJY and an NSF GRFP to KM.

\section{References}\label{references}

\setlength{\parindent}{-0.1in} \setlength{\leftskip}{0.125in} \noindent

\hypertarget{refs}{}
\hypertarget{ref-carpenter1998fourteen}{}
Carpenter, M., Akhtar, N., \& Tomasello, M. (1998). Fourteen-through
18-month-old infants differentially imitate intentional and accidental
actions. \emph{Infant Behavior and Development}, \emph{21}(2), 315--330.

\hypertarget{ref-castro2009human}{}
Castro, R. M., Kalish, C., Nowak, R., Qian, R., Rogers, T., \& Zhu, X.
(2009). Human active learning. In \emph{Advances in neural information
processing systems} (pp. 241--248).

\hypertarget{ref-coenen2017}{}
Coenen, A., Nelson, J. D., \& Gureckis, T. M. (2017). Asking the right
questions about human inquiry.

\hypertarget{ref-gabry2016rstanarm}{}
Gabry, J., \& Goodrich, B. (2016). Rstanarm: Bayesian applied regression
modeling via stan. r package version 2.10. 0.

\hypertarget{ref-goodman2016}{}
Goodman, N. D., \& Frank, M. C. (2016). Pragmatic language
interpretation as probabilistic inference. \emph{Trends in Cognitive
Sciences}, \emph{20}(11), 818--829.

\hypertarget{ref-grabinger1995rich}{}
Grabinger, R. S., \& Dunlap, J. C. (1995). Rich environments for active
learning: A definition. \emph{ALT-J}, \emph{3}(2), 5--34.

\hypertarget{ref-jara2016}{}
Jara-Ettinger, J., Gweon, H., Schulz, L. E., \& Tenenbaum, J. B. (2016).
The naïve utility calculus: Computational principles underlying
commonsense psychology. \emph{Trends in Cognitive Sciences},
\emph{20}(8), 589--604.

\hypertarget{ref-lee2014bayesian}{}
Lee, M. D., \& Wagenmakers, E.-J. (2014). \emph{Bayesian cognitive
modeling: A practical course}. Cambridge university press.

\hypertarget{ref-lindley1956}{}
Lindley, D. V. (1956). On a measure of the information provided by an
experiment. \emph{The Annals of Mathematical Statistics}, 986--1005.

\hypertarget{ref-mackay2003}{}
MacKay, D. J. (2003). \emph{Information theory, inference and learning
algorithms}. Cambridge university press.

\hypertarget{ref-nelson2005}{}
Nelson, J. D. (2005). Finding useful questions: On bayesian
diagnosticity, probability, impact, and information gain.
\emph{Psychological Review}, \emph{112}(4).

\hypertarget{ref-settles2012active}{}
Settles, B. (2012). Active learning. \emph{Synthesis Lectures on
Artificial Intelligence and Machine Learning}, \emph{6}(1), 1--114.

\hypertarget{ref-shafto2012learning}{}
Shafto, P., Goodman, N. D., \& Frank, M. C. (2012). Learning from
others: The consequences of psychological reasoning for human learning.
\emph{Perspectives on Psychological Science}, \emph{7}(4), 341--351.

\hypertarget{ref-sutton1998}{}
Sutton, R. S., \& Barto, A. G. (1998). \emph{Introduction to
reinforcement learning} (Vol. 135). MIT Press Cambridge.

\hypertarget{ref-xu2007}{}
Xu, F., \& Tenenbaum, J. B. (2007). Word learning as bayesian inference.
\emph{Psychological Review}, \emph{114}(2), 245.

\hypertarget{ref-yoon2017}{}
Yoon, E. J., Tessler, M. H., Goodman, N. D., \& Frank, M. C. (2017). ``I
won't lie, it wasn't amazing'': Modeling polite indirect speech. In
\emph{Proceedings of the thirty-ninth annual conference of the Cognitive
Science Society}.

\end{document}
